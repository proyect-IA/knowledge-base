\documentclass[letterpaper,11pt]{article}
\usepackage[utf8x]{inputenc}
\usepackage[bottom=1.6cm,top=1.6cm,left=2.2cm,right=2.6cm]{geometry}
\usepackage[spanish,mexico]{babel}
%\usepackage{listings}
\usepackage[many]{tcolorbox}
\tcbuselibrary{listings}
\usepackage{amsmath,amssymb,amsthm}
\usepackage{enumitem}
\setenumerate{itemsep=2pt,topsep=0pt, itemindent=4pt, labelindent=2pt}

\author{Edgar de Jesús Vázquez Silva}
\usepackage{listings}
\usepackage{color}
\usepackage{setspace}

\definecolor{dkgreen}{rgb}{0,0.7,0}
\definecolor{gray}{rgb}{0.5,0.5,0.5}
\definecolor{mauve}{rgb}{0.58,0,0.82}



\lstset{%frame=tb,
  language=Python,
  aboveskip=-2mm,
  belowskip=-2mm,
  showstringspaces=false,
  columns=flexible,
  basicstyle={\small\ttfamily},
  numbers=none,
  numberstyle=\tiny\color{gray},
  keywordstyle=\color{blue},
  commentstyle=\color{dkgreen},
  stringstyle=\color{mauve},
  breaklines=true,
  breakatwhitespace=true,
  tabsize=3,
  emph={ },
  emphstyle={\color{red}},
}

\newtcblisting{code}{
  listing only,
  %hbox,
  %colframe=cyan,
  colback=gray!10,
  %left skip=1pt,
  listing options={
   % frame=single,
    mathescape=true,
    numbers=left,
    numberstyle=\tiny\color{gray},
    numbersep=2pt,
  },
}

\renewcommand{\qedsymbol}{$\blacksquare$}
\addto\captionsspanish{\renewcommand\proofname{Prueba}}

\setlength{\parskip}{0cm}
\onehalfspacing

\begin{document}
\setlength{\parskip}{0cm}
\begin{center}
{\bf \large Proyecto 1: Knwledge Data Base}\\
Yoshio \\
Raul\\
Andric\\
Edgar de Jesús Vázquez Silva\\
Noviembre, 6,  2018\\
\vspace{0.2in}
\textbf{Inteligencia Artificial  2019-1}\\
Prof. Dr. Pineda\\
\end{center}

\vspace{\baselineskip}

Este proyecto consiste en construir una representación de la una base de conocimeinto,
en mundo abierto con implementación en Prolog.\\

Para la resolución de los requerimientos del proyecto se propone la siguiente metodología 
de una clase con las siguientes relaciones:\\

\begin{itemize}
  \item{nombre}
  \item{Padre}
  \item{Hijos}
  \item{Propiedades}
  \item{Relaciones}
  \item{Relaciones}
\end{itemize}

$class(nombre, padre, propiedades, relaciones, individios)$\\

\begin{itemize}
  \item{nombre: (átomo en prolog)}
  \item{Padre: (átomo en prolog)}
  \item{Propiedades: Lista en prolog, Propiedades:$[$     $]$, $atributo => valor$}
  \item{Relaciones}
  \item{Relaciones}
\end{itemize}

\end{document}
